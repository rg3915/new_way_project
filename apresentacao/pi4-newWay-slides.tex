\documentclass{fancyslides}

\usepackage[utf8]{inputenc}
\usepackage[T1]{fontenc}
\usepackage[brazil]{babel}
\usepackage{times}
\graphicspath{{figuras/}}

\usetheme{default} 
\setbeamertemplate{navigation symbols}{}
\setbeamercolor{structure}{fg=cyan} % Define the color of titles and fixed text elements (e.g. bullet points)
% \setbeamercolor{normal text}{fg=\yourowntexcol} % Define the color of text in the presentation
\setbeamercolor{normal text}{fg=\yourowntexcol} % Define the color of text in the presentation

% Define your own color as follows:
%\definecolor{pink}{rgb}{156,0,151}

\newcommand{\structureopacity}{0.75}

\newcommand{\strcolor}{blue}
\newcommand{\yourowntexcol}{white}

%----------------------------------------------------------------------------------------
%	TITLE SLIDE
%----------------------------------------------------------------------------------------

\newcommand{\titlephrase}{NEW WAY}
\newcommand{\subtitulo}{new-way.herokuapp.com} % Novos caminhos, novos horizontes
\newcommand{\name}{Régis da Silva\\ José Aciole\\ Gustavo Rodrigues\\ Rafael Carneo\\ Luiz Sousa Dias}
%Régis da Silva \& José Aciole}
\newcommand{\disciplina}{PI 4\textordmasculine Semestre}
\newcommand{\affil}{SENAC 2015}

\begin{document}

\fbckg[width=\paperwidth]{newway.jpg} % Slide background image
\startingslide % This command inserts the title slide as the first slide

%----------------------------------------------------------------------------------------
%	PRESENTATION SLIDES
%----------------------------------------------------------------------------------------

% \fbckg{05.jpg} % Slide background image
\begin{frame}
\tableofcontents
%\pointedsl{main point}
\end{frame}

%------------------------------------------------
\section{Planejamento do Projeto}
\fbckg{0005.jpg} % Slide background image
\begin{frame}\frametitle{Planejamento do Projeto}

\itemized{
	\item Régis: dba e desenvolvedor
	\item José: web design (layouts) e funcionalidades operacionais
	\item Gustavo: documentação
	\item Rafael: Testes e usabilidade
	\item Luiz: documentação
}

\end{frame}

%------------------------------------------------
\fbckg{05.jpg}
\begin{frame}\frametitle{Planejamento do Projeto}
\itemized{
	\item a partir dos requisitos montamos os mockups
	\item o \emph{modelo entidade-relacionamento}
	\item e as regras de negócio
}
\end{frame}


%------------------------------------------------
\fbckg[height=\paperheight]{new_way.png}
\begin{frame}\frametitle{\color{black}{Modelo Entidade-Relacionamento}}

\end{frame}

%------------------------------------------------
\section{Documentação Funcional Elaborada}
\fbckg{005.jpg}
\begin{frame}\frametitle{Documentação Funcional Elaborada}
\itemized{
	\item padrão de metodologia
	\item apresentação de estratégias de desenvolvimento
	\item produtividade
	\item manutenção
}
\end{frame}

%------------------------------------------------
\section{Modelo MTV}
\fbckg[height=\paperheight]{mtv.png}
\begin{frame}\frametitle{Modelo MVC}
	
\end{frame}

%------------------------------------------------
\section{Controle de Qualidade}
\fbckg[width=.7\paperwidth]{laudo.jpg}
\begin{frame}\frametitle{Controle de Qualidade}

\end{frame}

%------------------------------------------------
\section{Aplicação das práticas ITIL}
\fbckg{08.jpg}
\begin{frame}\frametitle{Aplicação das práticas ITIL}

\misc{O que precisa para que a aplicação funcione:
\begin{center}
	\begin{tabular}{r|l}
		Equipamentos	& Quiosque touch screen \\
		Arquitetura		& conexão com a internet \\
		Arquitetura		& servidor de rede \\
		SGBD			& PostgreSql \\
		Navegador		& Google Chrome \\
	\end{tabular}
\end{center}
}
\end{frame}

%------------------------------------------------
\section{Aplicação das práticas ITIL}
\fbckg{app.png}
\begin{frame}\frametitle{Aplicação das práticas ITIL}

\begin{center}
\color{blue}{O que foi usado na aplicação:

\

	\begin{tabular}{r|l}
		Linguagem		& Python 3.4.0 \\
		Framework		& Django 1.7.8 \\
		SGBD			& PostgreSql \\
		Front-end		& Html + Bootstrap \\
		Repository		& GitHub \\
		Deploy			& Heroku \\
	\end{tabular}
}
\end{center}
\end{frame}

%------------------------------------------------
\section{O Projeto}
\fbckg{05.jpg}
\begin{frame}\frametitle{O Projeto}
	\centering
	\vspace{-3cm}
	{\color{blue}{\LARGE NEW WAY}}

	\color{blue}{new-way.herokuapp.com}
\end{frame}

\end{document}